\section{绪论}\label{introduction}

本次毕业设计题目为 “软件变更分析系统的实现”,旨在实现一套可基于给定项目,依照不同版本间的软件变更,分析其对项目整体所能造成影响的系统。

软件维护旨在发现和解决软件项目中的漏洞,并且管理重要功能模块的软件变更。作为软件工程中一项关键过程,软件维护需要显著的时间与经济支出。\cite{SOLEIMANINEYSIANI2020106344}。现代软件的规模已有了显著的增长,对于当下的软件开发工程师而言,理解程序代码的变化是在团队中承担各种开发任务的根本,开发者需要理解软件所经历的变化、控制风险、减小软件代码漏洞,需要及时地对项目进行代码维护\cite{10.1145/2393596.2393656}。然而,软件规模的增长使得软件维护愈发困难。当下市场的大多数互联网公司都维护着庞大的内部代码库,这份代码库既是无数开发者智慧的结晶,同时也可能是会在任意时刻被引爆的火药。在逐步的迭代开发中,代码库可能出现诸如 “文档与项目不统一”、“背离设计初衷”、“代码质量降低” 等问题,这为各公司的开发部门带来了极大挑战。代码变更分析既作为软件维护流程中的关键,同时又是软件维护的关键难题之一,其主要聚焦于分析变更代码对存量代码的影响,借此降低开发团队在维护代码上的压力。

本文提出了一种简单的 “软件变更分析系统”,可用于一定条件下的软件变更分析。为使该项目具备一定的可扩展性,本次毕业设计选用了具备 “构建系统独特”、“编程语言繁杂”、“系统架构庞大” 的 AOSP (Android Open Source Project) 作为该系统展开软件分析的背景。由于 AOSP 具备上述三点分析上的困难,对其进行细致的分析工作十分艰难,因此该系统以 AOSP 下的 platform/frameworks 为例,主要围绕 “依照构建系统完成数据处理”、“变更获取”、“模块间分析” 以及 “模块内分析” 四点展开。

\begin{itemize}
    \item 在数据预处理方面,本系统希望能够获得对 AOSP 展开分析时必要的仓库模块关系以及元数据信息。
    \item 在变更获取方面,本系统希望能够使用仓库中的数据结构获得变更历史,并获得变更方法与变更类的相关信息。
    \item 在模块间分析方面,本系统希望能够根据给定遭到变更的源文件路径,获得该文件影响的所有文件路径。
    \item 在模块内分析方面,本系统希望根据变更方法,使用模块产物得到与变更方法相关的所有方法。
\end{itemize}

为获取不同版本间的软件变更,该系统实现了基于 git 版本控制系统的变更获取模块;为对系统展开模块间分析,本文扩展了 “有向图” 概念,通过将其应用到该系统依赖分析当中,实现了基于有向无环图的模块间分析功能;为对系统内部展开模块内分析,该系统实现了基于程序调用图的模块内分析功能。

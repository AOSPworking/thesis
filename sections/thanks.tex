\section*{谢辞}
\addcontentsline{toc}{section}{谢辞}

四年过去了。

其实没有特别大的变化。

2018 年 9 月 5 日上午,我和我爸妈站在校门口那个写着 “同济欢迎你” 的巨大展板前。那时的我绝对想不到四年过后的现在是怎样的。我眼中的世界在变化,我所处的社会在变化,我所在的学校在变化,我的家庭在变化,我的朋友们在变化,我也因此而变化。尽管一切都在不断地变化,我却还要说 “其实没有特别大的变化”。纷繁复杂花哨绚丽的事带不走我的一切,还不如迎面吹过的风。只不过,现在我正使用着我小学三年级就都认得的文字,写着我以往任一时刻都无法拼接出的词句段落。

毕业设计论文的致谢部分本不应如此沉重,但在这个时代背景下,我很难不用深沉的话语叙述我的过往、现在以及未来——过去四年里我通过各种途径学到了一些知识,这个过程十分曲折,我误入了很多不必要的陷阱。但我一直要求自己动用一切手段记住那些令我怄气、懊悔、愤恨、恼怒的事来提高我的精神耐受力,以便我未来再次遭遇它们时能做出更冷静的判断和更正确的解答。昨日种种,皆成今我,我做不到与过去的一切断绝联系。但与此同时,我正站在人生中某一阶段的最后关头,我想我还是有理由在此向我眼中的世界致谢。

在这不算成功的四年里,我受到了来自各方面的鼓励与支持。在这里,我首先要感谢我的家人们——感谢我的奶奶、我的爸爸以及我的爷爷。从小学到现在,我的家庭情况一直非常不好,这里需要尤其感谢奶奶对我的付出。虽说现在大学扩招严重,本科学历占总人口比例急剧攀升,但高考考上我校并写完毕业设计也算不辜负我奶奶对我的期望了。

接下来,我想感谢在这四年里提供给我帮助的老师们。这里我要感谢沈坚老师、陈宇飞老师、郭玉臣老师、卫志华老师、邓蓉老师。(以所学课程时间为序)感谢各位老师在课上课下对我的照顾。这里还要感谢我的研究生导师董震老师,其实拿到这个题目后,我一度以为自己无法完成,也有相当长的一段时间心情非常忧郁。好在老师给我提供了极为有效的指导,让我能够在正确的道路下不断前进。

此后,我想感谢现代大学制度。对我而言,大学最值得赞美的,莫过于它提供了一个结识同龄人的平台。感谢同济大学,你为我们提供了一个共同的话题。

因此,最后也是最重要的,我想感谢大学四年来为我提供了各方面帮助的各位同学!

首先我想感谢的是到了大学还与我有密切联系的高中同学们。这里我要感谢中央财经大学的雷诗雨同学、中国人民大学的饶迪同学、我校的孙汇通同学、大连理工大学的吴昊楠同学、北京大学的余舟{\CJKfontspec{宋体}飏}同学、重庆大学的张辰星同学、北京理工大学的周亮同学以及北京大学的周星宇同学。(以姓名拼音为序)感谢以上同学对我精神上的支持与帮助,感谢你们的理解与包容。

我还要感谢在本科四年中确实改变了我人生轨迹的同学们。这里我要着重感谢 17 级自动化专业的王怡琳同学、本专业的李培昊同学与吴子豪同学、17 级计科专业的刘佳伟同学、18 级计科专业的鞠璇同学与赵中楷同学、16 级计科专业的颜正辉同学以及我校软件学院 18 级的刘雪迪同学。(以影响我的时间前后为序)与从真实世界获得的感悟相比,这四年里我在课堂与书本上得到的知识极为有限。我十分感激在我校遇到了许多能改变我自己的人。

接下来,我要感谢我的室友们,感谢 18 级计科专业的郭嘉胥同学、纪宇同学、李宇龙同学、杨宏辉同学以及 18 级信安专业的甘源同学、胡新邦同学、许帅同学。(以时间与姓名拼音为序)室友是我本科四年里陪伴我时间最久的人,在这里非常感谢各位室友对我的照顾。

最后我想感谢这四年来同济大学里每一个和我展开过一定强度的线上闲聊的同学。再次感谢上面提到的王怡琳同学、李培昊同学、刘雪迪同学,感谢 18 级计科专业的俞少作同学、胡行健同学、张海同学、朱姝{\CJKfontspec{宋体}玥}同学、张晨阳同学,感谢 18 级自动化专业的陈旭海同学,感谢 18 级大数据专业的段抒彤同学、卞逸凡同学,感谢 19 级计科专业的周珂帆同学、彭斐然同学,感谢 19 级信安专业的商睿同学。(受篇幅限制,还望未被提及的同学不要介意)感谢这些同学在我孤独寂寞时与我聊天,愿过去的美好在过去永远美好。

本科四年过去了。我的毕业没有合照,没有聚餐,没有旅行。我拥有的只是一次匆忙的离去——而我带不走无奈散落在我桌前的许多回忆,以及一次又一次唐突的离别——而我事先并未知晓它们都是最后一次相见。我眼中的世界正遭受着莫大的考验,而我对此无能为力。

受制于篇幅,我很难在短短的一个章节内完全表露我此刻心中的所有感激之情。

感谢各位,祝这世界中的我毕业快乐。
